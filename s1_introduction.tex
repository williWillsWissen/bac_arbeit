%----------------------------------------------
% Introduction
%----------------------------------------------
\section{Introduction}
\label{sec:introduction}

%----------------------------------------------
% * Motivation 
%----------------------------------------------
\subsection{Motivation}
\label{sec:Motivation}

%----------------------------------------------
% ** Interaktion mit einer Bildsuchmaschiene
%----------------------------------------------
\subsubsection{Interface für Kreshmoi}
\label{sec:Interface für Kreshmoi}
Das Ziel von KHRESMOI ist das Durchsuchen und der Zugang zu medizinischen Informationen für verschiedene Benutzergruppen mit unterschiedlichem medizinischen Vorwissen.
Die Einteilung der Benutzer erfolgt in 3 Kategorien:
\begin{itemize}
	\item Personen ohne speziellen medizinischen Kentnissen
	\item Ärzte
	\item Radiologen
\end{itemize}
Dazu verknüpft KHRESMOI Daten aus verschiedenen heterogenen Resourcen wie Bildern aus PACS, Bildern und Text aus Publikationen in Journalen oder Daten von Webseiten.
Da sich die verschiedenen Resoucen qualitativ sehr stark voneinander unterscheiden können wird einen Bewertung ihrer Glaubwürdigkeit durchgeführt und dem Benutzer angezeigt.
\\
Die Suchanfrage kann in textueller Form oder als Bild-Query sowie als Kombination von beidem gestellt werden.
Ein weiteres wichtiges Feature hiebei ist die multilinguale Suche, da die Menge an verfügbaren medizinischen Informationen nicht in alle Sprachen gleich ist.
Dies bedeutet dass die Suchanfrage in mehrere Sprachen übersetzt wird und somit auch anderssprachige Quellen durchsucht werden können.
Die Zusammenfassungen der Suchergebnisse werden anschließend in die Anfragesprache rückübersetzt wodurch der Benutzer schnell duch die Ergebnissliste navigieren kann.

%----------------------------------------------
% ** Interaktion mit einer Bildsuchmaschiene
%----------------------------------------------
\subsubsection{Interaktion mit einer Bildsuchmaschiene}
\label{sec:Interaktion mit einer Bildsuchmaschiene}
Ein Teilprojekt von KHRESHMOI ist das Druchsuchen von medizinischen Bilddaten wobei diese in 2D, 3D oder 4D (Video) vorliegen können.
Um eine Suchanfrage auf ein Bild stellen zu können müssen an einem Referenz-Bild Bereiche eingezeichnet werden welche dann die Anfrage formen.
Aus der Textur eines markierten Bereiches wird ein Feature-Vektor extrahiert mit dem anschließend eine Datenbank von zuvor indizierten Bildern durchsucht wird.
\\
Ein Frontend einer Bildsuchmaschiene muss daher sowohl Tools zum markieren von interessanten Bereichen, 
als auch die Funktionalität zur verfnünftigen Berachtung der Bilder bereitstellen.
\\
Diese Arbeit spezialisiert sich auf das Durchsuchen von radioloischen Aufnahmen in 2D und 3D welche in einem PACS (Picture Archiving and Communication Systems) abgelegt sind.
Da in einem Krankenhaus täglich große Mengen an Daten durch radionlogische Aufnahmen produziert werden, 
bietet eine effizientes Durchsuchen dieser die möglichkeit Sie für Ausbildung und Forschug wieder zu verwenden.
\\
Dazu muss das User-Interface die grundliegenden Funktionen eines Betrachtungs-Tools für Röngten- und Computertomographie-Aufnahmen zur verfüfung Stellen:
\begin{itemize}
	\item Zoom
	\item Schnelles anpassen von Kontrast und Helligkeit
	\item Navigation durch die Schnitte eins 3D-Körpers in einer Schnittachse
\end{itemize}


%----------------------------------------------
% ** Hardware und OS Abrstraktion
%----------------------------------------------
\subsubsection{Hardware und OS Abstraktion}
\label{sec:Hardware und OS Abstraktion}

%----------------------------------------------
% * Pflichtenheft
%----------------------------------------------
\subsection{Pflichtenheft}
\label{sec:Pflichtenheft}

%----------------------------------------------
% * Möglichkeiten zu Umsetzung
%----------------------------------------------
\subsection{Möglichkeiten zur Umsetzung}
\label{sec:Möglichkeiten zur Umsetzung}

%----------------------------------------------
% ** JavaApplet
%----------------------------------------------
\subsubsection{JavaApplet}
\label{sec:JavaApplet}

%----------------------------------------------
% ** HTML5
%----------------------------------------------
\subsubsection{HTML5}
\label{sec:HTML5}



