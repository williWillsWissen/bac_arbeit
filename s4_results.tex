%!TEX root = cvl_bachelor_thesis.tex

%-------------------------------------------------------
% Results
%-------------------------------------------------------
\section{Results}
\label{sec:results}

%----------------------------------------------
% * Geschwindigkeit 
%----------------------------------------------
\subsection{Performance}
\label{sec:Performance}

In Absatz \ref{sec:Berehnung_der_Fensterung} wurden zwei Ansätze zur Berechnung der Fensterung vorgestellt,
welche jeweils auf unterschiedlichen vom Browser zur verfügung gestellten API's basieren.
%TODO: Introducation

\subsubsection{Messung}
Um zu evaluieren wie performant ein Verfahren ist,
wurde über einen bestimmten Zeitraum $T_m$ ermittelt, wie viele Bilder durchschnittlich in der Sekunde berechnet werden können.
Dieser Wert wird als Framerate bezeichnet und in FPS (Frames per Second) angegeben.
Die Framerate ist von folgenden Faktoren abhängig:
\begin{itemize}
	\item Berechnungs-Verfahren
	\item Anzahl der Pixel im Bild
	\item Verwendete Hardware
	\item Verwendeter Browser
	\item Aktuelle Auslatung der Maschine
\end{itemize}
Die ersten vier Parameter wurden für die verschiedenen Messungen variert, während der letzte eine Störgröße dartstellt welche nur bedingt beeinflusst und nicht absolut konstant gehalten werden kann.
Bei jeder Messung wurden die durchschnittlichen FPS in Abhängigkeit von der Bildgröße aufgenommen.
Eine Messung besteht daher aus einer Menge Messpunkten wobei jeder die durchschnittliche Framerate für eine bestimmte Bildgröße angiebt.
Dabei wurde eine quadratisches Bild verwendet und dessen Kantenlänge zwischen zwei Messpunkten jeweils um einen bestimmte Wert $\Delta_{edge}$ vergrößert.
Die Pixelanzahl zwischen zwei Messpunkten hat somit ein quadratisches Wachstum wie es auch beim Zoomen von Bildern in der Applikation der Fall ist.

Für die Durchführung der Messung wurde eine eigene Messumgebung entwickelt mit welcher es möglich ist die Implementierung der beiden Verfahren mit unterschliedlichen Parametern zu testen.
Die Messumgebung besteht aus einem Servlet welches mit Spring MVC implementiert wurde und aus einer Clientseitigen Testumgebung für die Implementierungen der beiden Verfahren.
Eine Messung wird durch den Aufruf einer Resource auf dem Servlet gestartet und parametriert.
Das Servlet retourniert darauf hin eine Webseite in welcher die Testumgebung als JavaScript ausgeführt wird.
Diese versucht auf einem Bild die Fensterung mit einer Ziel-Framerate von 60FPS zu berechnen und misst Abfälle in der Framerate.
Zur Messung wurde das Tool Stats.JS verwendet und geringfügig modifiziert um die gemessenen Werte auslesen zu können.
Nach ablauf der Messzeit $T_m$ werden die gemessenen Werte zurück an den Server gesendet welcher diese in eine Datenbank schreibt.
Der Server antwortet darauf mit einem neuen Messpunkt wobei das Testbild zwischen den Messpunkten jedes mal vergrößert wird bis eine vorher festgelegte maximale Größe ereicht ist.


%TODO: Framework
%TODO: Tests
%TODO: Results
%TODO: Interpretation

%----------------------------------------------
% * Usability
%----------------------------------------------
\subsection{Usability}
\label{sec:Usability}
%markus fragen
