%----------------------------------------------
% Related Work
%----------------------------------------------
\section{Related Work}
\label{sec:relatedWork}
Software für die Bildsuche in radiologischen Daten gibt es bis jetzt noch nicht, 
allerdings decken sich die Anforderungen großteils mit Betrachtungstools für 2D und 3D Daten aus der Radiologie und Nuklearmedizin.
Solche Softwareprodukte finden sich in den Betrachtungs-Workstations von PACS-System in Krankenhäusern oder als Betrachtungstools für Datensätze des offenen Standards DICOM.
\\
Die beiden Konzepte PACS und DICOM werden im folgenden Kapitel kurz erklärt, sowie die Umsetzung der benötigten Funktionalität in zwei konkreten Softwareproduken diskutiert.

%----------------------------------------------
% * PACS-Systeme
%----------------------------------------------
\subsection{PACS-Systeme}
\label{sec:PACS-Systeme}
Ein PACS-System (Picture Archiving and Communication System) dient zum Speichern und Austausch von medizinischen Bilddaten.
Obwohl es prinzipel für alle bildgebenden Verfahren verwendet werden kann wird es vorwiegend für Daten aus der Radiologie und Nuklearmedizien genutzt.
\\
Das Systems setzt sich aus dem PACS Server und den Workstations zusammen.
Der Server sammelt Daten von den bildgebenden Geräten,
 verknüpft Sie mit Daten aus einem Krankenhaus Informations System (KIS) oder Radiologie Informations System (RIS) und sorgt für ihre Archivierung in einem Kurz oder Langzeitarchiv. 
Die Kommunikation mit den bildgebenden Geräten erfolgt meist durch ein Protokoll welches den DICOM Standard implementetiert.
\\
Die Befundung erfolgt auf den PACS Workstations welche die Daten vom Server laden und anzeigen.
Die Workstation stellte die Funktionalität zur Betrachtung und zum Nachbearbeiten der Bilder zur verfügung.
Änderungen der Daten werden von der Workstation zurück auf den Server geladen.
Je nach Funktionsumfang stehen auch Tools zur Befundung zur verfügung welche die Daten and das RIS oder KIS weiter geben.
%TODO: cite pacs - seite 219-224
%TODO: Wie gehören Seiten zitiert

%----------------------------------------------
% * DICOM
%----------------------------------------------
\subsection{DICOM}
\label{sec:DICOM}
DICOM steht für Digital Imaging and Communication in Medicine und ist ein offener Standart welcher die Übertragung und das Speichern von medizinischer Bildinformation spezifiziert.
Die wesentlichen Teile der Spezifikation sind die Datenstruktur für die Bilddaten und zugehörige Metainformationen, 
Services welche auf diesen Daten operieren, Anforderung an DICOM conforme Hard- und Software-Produkte und das Ablegen der Informationen auf einem Datenträger.
Das Datenmodel setzt sich in Anlehnung an die reale Welt grundliegend aus den Entitäten Patient, Studie, Serie und Image zusammen zwischen denen jeweils eine 1:n oder 0:n Beziehung besteht.
Es bietet weiters ausreichen Möglichkeiten zur Erweiterung durch einen Definitionsmechnaissum für alle DICOM Objekte die sogenannte Image Object Defintion.

%----------------------------------------------
% * 
%----------------------------------------------
\subsection{XYZ - PACS Workstation}
\label{sec:XYZ - PACS Workstation}


%----------------------------------------------
% * Osirix
%----------------------------------------------
\subsection{Osirix}
\label{sec:Osirix}
OsiriX ist eine Software zur Betrachtung und Nachbearbeitung von DICOM Bildaten.
Sie wird als freie Open-Source-Software unter der GPL für das Betriebssystem Mac OS X entwickelt.
OsiriX ist nur für die Forschung und dem privaten gebrauch zugelassen, 
für einen einen diagnostischen Einsatz in der Medizin steht die kostenpflichtige Version OsiriX MD zur verfügung.
\\
Das Programm führt eine Datenbank von DICOM Datensätzen, 
welche von DICOM-Datein importiert bzw auch wieder als solche exportiert werden können.
Weiters können über das DICOM-Protokoll die Daten auch von einem PACS-Server geladen werden.


\subsection{Funktionsumfang von Betrachtungstools}
\label{sec:Funktionsumfang von Betrachtungstools}

\subsubsection{Zoom und Scroll}
\subsubsection{Fensterung und Level}
\subsubsection{Histogram Modifikation}
\subsubsection{Negativ bilden}
\subsubsection{Distanz und Flächenmaß}
\subsubsection{Collagen}
%TODO: Kontakt markus .... was gibst schon
%TODO: Kontakt markus .... pacs workstations

