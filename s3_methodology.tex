%-------------------------------------------------------
% Methodology
%-------------------------------------------------------
\section{Methodology}
\label{sec:methodology}

%----------------------------------------------
% * Verwendetet Technologien
%----------------------------------------------
\subsection{Verwendetet Technologien und Protokolle}
\label{sec:Verwendetet Technologien}

%----------------------------------------------
% ** HTTP
%----------------------------------------------
\subsubsection{HTTP}
\label{sec:HTTP}

%----------------------------------------------
% ** REST
%----------------------------------------------
\subsubsection{REST}
\label{sec:HTTP}

%----------------------------------------------
% ** JSON
%----------------------------------------------
\subsubsection{JSON}
\label{sec:JSON}

%----------------------------------------------
% ** AJAX
%----------------------------------------------
\subsubsection{AJAX}
\label{sec:AJAX}

%----------------------------------------------
% ** Objectiv J
%----------------------------------------------
\subsubsection{Objectiv J}
\label{sec:Objectiv J}

Objective J ist eine Programmiersprache welche sich von der Syntax stark an Objective C anlehnt.
Sie ist eine Erweiterung oder Obermenge von Javascript und wird von einem in Javascript geschriebenen Interpreter abgearbeitet.
In Javascript können Objekte durch Prototyping erstellt werden, das Konzept von Klassen wird aber nicht unterstüzt.
Obj J bietet zusätzlich zu den nativen JS Objekten die definition von Klassen inklusive Vererbung und die generierung von Objekten daraus.
Obwohl es die Sprache erlaubt für Variablen, Methodenparameter und Rückgaben eine Datentyp zu definieren, 
werden diese aufgrund von schwacher Typisierung vom Interpreter nicht auf ihre Einhaltung überprüft.
In der aktullen Version wird die Übergabe von Referenzen als Parameter ähnlich einem Pointer in C unterstützt.
\cite{capp}

%----------------------------------------------
% ** Cappuccino
%----------------------------------------------
\subsubsection{Cappuccino}
\label{sec:Cappuccino}

%----------------------------------------------
% * Kommunikation mit KRESHMOI
%----------------------------------------------
\subsection{Kommunikation mit KRESHMOI}
\label{sec:Kommunikation mit KRESHMOI}
%Noch in der entwicklung, am besten später machen

%----------------------------------------------
% ** Query nach Bildern
%----------------------------------------------
\subsubsection{Query nach Bildern}
\label{sec:Query nach Bildern}

%----------------------------------------------
% ** Laden der Bilder
%----------------------------------------------
\subsubsection{Laden der Bilder}
\label{sec:Laden der Bilder}


%----------------------------------------------
% * Architektur und Komponenten
%----------------------------------------------
\subsection{Architektur und Komponenten}
\label{sec:Architektur und Komponenten}

%----------------------------------------------
% ** Domänen Model
%----------------------------------------------
\subsubsection{Domänen Model}
\label{sec:Domänen Model}

%----------------------------------------------
% ** Architektur und Aufteilung in Komponenten
%----------------------------------------------
\subsubsection{Architektur und Aufteilung in Komponenten}
\label{sec:Architektur und Aufteilung in Komponenten}

%----------------------------------------------
% ** 2DView
%----------------------------------------------
\subsubsection{2DView}
\label{sec:2DView}

%----------------------------------------------
% ** Kommunikations Module
%----------------------------------------------
\subsubsection{Kommunikations Module}
\label{sec:Kommunikations Module}

%----------------------------------------------
% * Usability
%----------------------------------------------
\subsection{Usability}
\label{sec:Usability}

%----------------------------------------------
% ** Workflow bei der Befundung
%----------------------------------------------
\subsubsection{Workflow bei der Befundung}
\label{sec:Workflow bei der Befundung}

