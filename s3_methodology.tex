%-------------------------------------------------------
% Methodology
%-------------------------------------------------------
\section{Methodology}
\label{sec:methodology}

%----------------------------------------------
% * Verwendetet Technologien
%----------------------------------------------
\subsection{Verwendetet Technologien und Protokolle}
\label{sec:Verwendetet Technologien}

%----------------------------------------------
% ** HTTP
%----------------------------------------------
\subsubsection{HTTP}
\label{sec:HTTP}

HTTP (Hyper Text Transfer Protokoll) ist ein Protokoll zur übertragung von Daten über ein Netzwerk welches auf TCP aufsetzt.
Der Datenaustausch zwischen zwei Kommunikationspartner findet in der From von Nachrichten statt, 
wobei der Client eine Anfrage an einen Server stellt und dieser die Anfrage bearbeitet und eine Antwort retuniert.
\\
Eine Nachricht setzt sich aus einem Header und einen Body zusammen.
Der Body enthält die Nutzdaten und der Header enthält Metadaten über die Nutzdaten.
Vom Aufbau der Nachricht unterscheiden sich Anfrage und Antwort nur in der ersten Zeile:
\begin{itemize}
	\item Anfrage: Enthält die HTTP-Methode, die URL welche auf die Resource am Server zeigt und die Protokollversion.
	\item Antwort: Enthält die Protokollversion und den Serverstatus. 
		Der Serverstatus liefert eine Aussage ob der Request erfolgreich bearbeitet wurde bzw welche Art von Fehler bei der Bearbeitung aufgetreten ist.
\end{itemize}
HTTP ist ein zustandsloses Protokoll, daher wird nach jeder Anfrage die Verbindung vom Server wieder abgebaut.
Für eine Zuordnung eines Clients muss dieser eine Session-ID mitsenden welche normalerweise im Header enthalten ist.
% TODO: RFC zietieren

%--------------------------------------------- 
% ** REST
%----------------------------------------------
\subsubsection{REST}
\label{sec:HTTP}
REST ist im eigentlichen Sin mehr ein Architekturstiel als ein Protokoll welcher mit HTTP umgesetzt wird.
Die Idee von REST ist dass eine URL genau eine Resource auf einem Server addressiert, 
wobei eine Resource eine statische Datei oder das ergebniss einer aktion auf dem Server sein kann.
\\
Der Architekturstiel ist lässt sich durch fünf Prizipien zusammenfassen:
\\
\\
\textbf{Resource mit eindeutiger Identifikation:}\\
Jede Resrouce wird durch eine URI (Uniform Resouce Identifier) weltweit eindeutig identifiziert.
Diese addressiert unter anderem den Server auf den sich die resource befindet sowie Resource auf dem Server selbst.
\\
\\
\textbf{Hypermedia}\\
Verknüfungen zu anderen Entitäten und werden als Links auf die jeweiligen Resourcen dargestellt.
Weiters kann die Steuertung des Applikationszustandes durch Links auf weiter Aktionen durch Hypermedia umgesetzt werden.
\\
\\
\textbf{Standard-Opperationen}\\
Es gibt ein definiertes Interface welches von jeder Resource zur verfügung gestellt werden muss.
Dieses umfasst einen relativ kleinen Satz von Opperationen welche auf die Resource ausgeführt werden können.
\\
\\
\textbf{Unterschiedliche repräsentation der Resourcen}\\
Die Resourcen können unterschiedliche Darstellungsformen haben.
Ein Client kann also eine Resource in einem bestimmten Format (z.B.: XML, HTML, JSON) anfordern sofern diese Darstellung vom Server unterstüzt wird.
In HTTP wird die gewünschte Dartstellung im Header angegeben.
\\
\\
\textbf{Zustandslose Kommunikation}\\
Der Server hällt keine Zustandsinformationen über den Client welcher über die Dauer eines Requests hinaus geht.
Daher muss der Zustand einer Anwendung entweder am Client liegen oder vom Server in eine Resource umgewandelt werden.
%----------------------------------------------
% ** JSON
%----------------------------------------------
\subsubsection{JSON}
\label{sec:JSON}

%----------------------------------------------
% ** AJAX
%----------------------------------------------
\subsubsection{AJAX}
\label{sec:AJAX}
AJAX (Asynchronous JavaScript and XML) ermöglicht es einer Webanwendung kleinere Mengen von Daten nachzuladen und damit Teile der Webseite dynamisch zu ändern, 
statt bei jeder Aktion die Webseite neu zu laden.
\\
Benötigt die Web Appliketion Daten vom Server wird an diesem eine HTTP Anfrage gesendet und Callback-Funktionen für den Fall einer Antwort oder eines Fehlers beim Browser registriert.
Erhällt der Browser eine Antwort auf sine Anfrage ruft er die Callback-Funktion auf und übergibt die erhalten Daten wodurch die Webanwendung mit der Verarbeitung dieser fortfahren kann.
\\
Dies ermöglicht die Entwicklung komplexer Webapplikationen, wobei die Webapplikation selbst mit der Seite geladen wird und die Daten die der Benutzer mit der Anwendung verarbeiten möchte dynamisch von dieser nachgeladen werden können.

%----------------------------------------------
% ** Objectiv J
%----------------------------------------------
\subsubsection{Objectiv J}
\label{sec:Objectiv J}

Objective J ist eine Programmiersprache welche sich von der Syntax stark an Objective C anlehnt.
Sie ist eine Erweiterung oder Obermenge von Javascript und wird von einem in Javascript geschriebenen Interpreter abgearbeitet.
In Javascript können Objekte durch Prototyping erstellt werden, das Konzept von Klassen wird aber nicht unterstüzt.
Obj J bietet zusätzlich zu den nativen JS Objekten die definition von Klassen inklusive Vererbung und die generierung von Objekten daraus.
Obwohl es die Sprache erlaubt für Variablen, Methodenparameter und Rückgaben eine Datentyp zu definieren, 
werden diese aufgrund von schwacher Typisierung vom Interpreter nicht auf ihre Einhaltung überprüft.
In der aktullen Version wird die Übergabe von Referenzen als Parameter ähnlich einem Pointer in C unterstützt.
\cite{capp}

%----------------------------------------------
% ** Cappuccino
%----------------------------------------------
\subsubsection{Cappuccino}
\label{sec:Cappuccino}
Bei Cappuccino handelt es sich um ein Web Application Framework für Objectiv J und Javascript, welches haupsächlich der Erstellung komplexer Benutzeroberflächen dient.
Das Framework lehnt sich sowol vom Aussehen als auch von der Benennung der Komponenten sehr stark an das GUI-Framework Cocoa von Apple an.
GUI-Elemente werden als Objekte erstellt welche von einer View-Klasse erben und innerhalb von anderen Views positioniert werden können.
Das Interface wird von einem HTML5 fähigen Browser gerendert wobei für dessen Erstellung keinerlei HTML oder CSS kentnisse notwendig sind.

%----------------------------------------------
% * Kommunikation mit KRESHMOI
%----------------------------------------------
\subsection{Kommunikation mit KRESHMOI}
\label{sec:Kommunikation mit KRESHMOI}
%Noch in der entwicklung, am besten später machen

%----------------------------------------------
% ** Query nach Bildern
%----------------------------------------------
\subsubsection{Query nach Bildern}
\label{sec:Query nach Bildern}

%----------------------------------------------
% ** Laden der Bilder
%----------------------------------------------
\subsubsection{Laden der Bilder}
\label{sec:Laden der Bilder}


%----------------------------------------------
% * Architektur und Komponenten
%----------------------------------------------
\subsection{Architektur und Komponenten}
\label{sec:Architektur und Komponenten}

%----------------------------------------------
% ** Domänen Model
%----------------------------------------------
\subsubsection{Domänen Model}
\label{sec:Domänen Model}

%----------------------------------------------
% ** Architektur und Aufteilung in Komponenten
%----------------------------------------------
\subsubsection{Architektur und Aufteilung in Komponenten}
\label{sec:Architektur und Aufteilung in Komponenten}

%----------------------------------------------
% ** 2DView
%----------------------------------------------
\subsubsection{2DView}
\label{sec:2DView}

%----------------------------------------------
% ** Kommunikations Module
%----------------------------------------------
\subsubsection{Kommunikations Module}
\label{sec:Kommunikations Module}

%----------------------------------------------
% * Usability
%----------------------------------------------
\subsection{Usability}
\label{sec:Usability}

%----------------------------------------------
% ** Workflow bei der Befundung
%----------------------------------------------
\subsubsection{Workflow bei der Befundung}
\label{sec:Workflow bei der Befundung}

